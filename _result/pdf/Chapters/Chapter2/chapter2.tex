% -*- mode: latex; -*- mustache tags:  
\documentclass[10pt,twoside,english]{_support/latex/sbabook/sbabook}
\let\wholebook=\relax

\usepackage{import}
\subimport{_support/latex/}{common.tex}

%=================================================================
% Debug packages for page layout and overfull lines
% Remove the showtrims document option before printing
\ifshowtrims
  \usepackage{showframe}
  \usepackage[color=magenta,width=5mm]{_support/latex/overcolored}
\fi


% =================================================================
\title{Object-Centric Instrumentation with Pharo}
\author{Steven Costiou}
\series{Square Bracket tutorials}

\hypersetup{
  pdftitle = {Object-Centric Instrumentation with Pharo},
  pdfauthor = {Steven Costiou},
  pdfkeywords = {object-centric, meta-programming}
}


% =================================================================
\begin{document}

% Title page and colophon on verso
\maketitle
\pagestyle{titlingpage}
\thispagestyle{titlingpage} % \pagestyle does not work on the first one…

\cleartoverso
{\small

  Copyright 2017 by Steven Costiou.

  The contents of this book are protected under the Creative Commons
  Attribution-ShareAlike 3.0 Unported license.

  You are \textbf{free}:
  \begin{itemize}
  \item to \textbf{Share}: to copy, distribute and transmit the work,
  \item to \textbf{Remix}: to adapt the work,
  \end{itemize}

  Under the following conditions:
  \begin{description}
  \item[Attribution.] You must attribute the work in the manner specified by the
    author or licensor (but not in any way that suggests that they endorse you
    or your use of the work).
  \item[Share Alike.] If you alter, transform, or build upon this work, you may
    distribute the resulting work only under the same, similar or a compatible
    license.
  \end{description}

  For any reuse or distribution, you must make clear to others the
  license terms of this work. The best way to do this is with a link to
  this web page: \\
  \url{http://creativecommons.org/licenses/by-sa/3.0/}

  Any of the above conditions can be waived if you get permission from
  the copyright holder. Nothing in this license impairs or restricts the
  author's moral rights.

  \begin{center}
    \includegraphics[width=0.2\textwidth]{_support/latex/sbabook/CreativeCommons-BY-SA.pdf}
  \end{center}

  Your fair dealing and other rights are in no way affected by the
  above. This is a human-readable summary of the Legal Code (the full
  license): \\
  \url{http://creativecommons.org/licenses/by-sa/3.0/legalcode}

  \vfill

  % Publication info would go here (publisher, ISBN, cover design…)
  Layout and typography based on the \textcode{sbabook} \LaTeX{} class by Damien
  Pollet.
}


\frontmatter
\pagestyle{plain}

\tableofcontents*
\clearpage\listoffigures

\mainmatter

\chapter{What we are talking about}
In the next chapters, we evaluate each technique following a three-fold evaluation. First, the studied technique is applied on a simple example of object-centric instrumentation. Second, the technique is evaluated against a set of desirable properties. Finally, performance overhead are evaluated. Only the raw solution is evaluated, without considering the possibility of enhancing the technique by building something on top.
\section{Illustration example}
Each solution is experimented on a trivial example of object-centric behavior instrumentation. This example is illustrated in script \ref{motivation-example}. Two instances of \textcode{OrderedCollection} are created, and to each of these instances is sent the \textcode{add:} message with a string as a parameter. The instrumentation must happen in-between. We want to instrument the \textcode{col1} object, so that when the \textcode{add:} message is received, the size of the collection and the added object (passed as parameter) are printed in the \textcode{Transcript}.

\begin{listing}[float, label=motivation-example]{smalltalk}{Trivial example for object-centric instrumentation}
|col1 col2|
col1 := OrderedCollection new.
col2 := OrderedCollection new.

"...instrumentation must happen here..."

col1 add: 'Hello World'.
col2 add: 'Hello World'.
\end{listing}
\section{Evaluation criteria}
Each solution is evaluated agains the following desirable properties.

\begin{tabular}{ll}
\toprule
\textbf{Property} & \textbf{Definition} \\
\midrule
Manipulated entity & The unit of instrumentation \\
 & (\textit{e.g.} a class, a Trait, an object...) \\
Reusability & The entity can be reused to instrument different objects \\
Flexibility & Instrumentation does not put constraint on the \\
 & source code or in the coding style \\
Granularity & The level of at which behavior can be instrumented \\
 & (\textit{e.g.} method, AST...) \\
Integration & Instrumentation does not break system features \\
\bottomrule
\end{tabular}
\section{Performance overhead evaluation}
\begin{itemize}
\item source code with instrumentation example
\item describe the used method
\item describe the limitations of the method
\end{itemize}



\bibliographystyle{alpha}
\bibliography{book.bib}

% lulu requires an empty page at the end. That's why I'm using
% \backmatter here.
\backmatter

% Index would go here

\end{document}
