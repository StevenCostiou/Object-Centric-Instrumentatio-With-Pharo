% -*- mode: latex; -*- mustache tags:  
\documentclass[10pt,twoside,english]{_support/latex/sbabook/sbabook}
\let\wholebook=\relax

\usepackage{import}
\subimport{_support/latex/}{common.tex}

%=================================================================
% Debug packages for page layout and overfull lines
% Remove the showtrims document option before printing
\ifshowtrims
  \usepackage{showframe}
  \usepackage[color=magenta,width=5mm]{_support/latex/overcolored}
\fi


% =================================================================
\title{Object-Centric Instrumentation with Pharo}
\author{Steven Costiou}
\series{Square Bracket tutorials}

\hypersetup{
  pdftitle = {Object-Centric Instrumentation with Pharo},
  pdfauthor = {Steven Costiou},
  pdfkeywords = {object-centric, meta-programming}
}


% =================================================================
\begin{document}

% Title page and colophon on verso
\maketitle
\pagestyle{titlingpage}
\thispagestyle{titlingpage} % \pagestyle does not work on the first one…

\cleartoverso
{\small

  Copyright 2017 by Steven Costiou.

  The contents of this book are protected under the Creative Commons
  Attribution-ShareAlike 3.0 Unported license.

  You are \textbf{free}:
  \begin{itemize}
  \item to \textbf{Share}: to copy, distribute and transmit the work,
  \item to \textbf{Remix}: to adapt the work,
  \end{itemize}

  Under the following conditions:
  \begin{description}
  \item[Attribution.] You must attribute the work in the manner specified by the
    author or licensor (but not in any way that suggests that they endorse you
    or your use of the work).
  \item[Share Alike.] If you alter, transform, or build upon this work, you may
    distribute the resulting work only under the same, similar or a compatible
    license.
  \end{description}

  For any reuse or distribution, you must make clear to others the
  license terms of this work. The best way to do this is with a link to
  this web page: \\
  \url{http://creativecommons.org/licenses/by-sa/3.0/}

  Any of the above conditions can be waived if you get permission from
  the copyright holder. Nothing in this license impairs or restricts the
  author's moral rights.

  \begin{center}
    \includegraphics[width=0.2\textwidth]{_support/latex/sbabook/CreativeCommons-BY-SA.pdf}
  \end{center}

  Your fair dealing and other rights are in no way affected by the
  above. This is a human-readable summary of the Legal Code (the full
  license): \\
  \url{http://creativecommons.org/licenses/by-sa/3.0/legalcode}

  \vfill

  % Publication info would go here (publisher, ISBN, cover design…)
  Layout and typography based on the \textcode{sbabook} \LaTeX{} class by Damien
  Pollet.
}


\frontmatter
\pagestyle{plain}

\tableofcontents*
\clearpage\listoffigures

\mainmatter

\chapter{Anonymous subclasses}
Anonymous classes are nameless classes that are inserted between an object and its original class \cite{foote1989reflective,hinkle1993debugging}. The object is migrated to that new class, which takes the original object's class as its superclass. Methods from the original class can be redefined and reimplemented in the anonymous class, having the effect to change the behavior of that single object. Original behavior that is not redefined in the anonymous subclass is preserved. It is one of the fastest implementation for object-centric instrumentation \cite{ducasse1999evaluating}.
\section{Example}
Anonymous subclasses are derived from the original class of the object (line 3). Methods must be manually (re)written with instrumentation and compiled in the new class (line 4-8). Then the object has to be migrated to its new class (line 10). To rollback the instrumentation, the object must be manually migrated back to its original class (line 12). The migration is not \textit{safe} if more than one process is using the instrumented object.

\begin{displaycode}{plain}
|person anonClass|
  person := Person new.
	anonClass := anObject class newAnonymousSubclass.
	anonClass
		compile:
			'name: aName
				self tag: aName.
				name := aName'.
  "migrates the object to its new class"
	anonClass adoptInstance: person.
  "migrates back the object to its original class"
  anonClass superclass adoptInstance: person.
\end{displaycode}
\section{Evaluation}
\textbf{Manipulated entity: Classes.} Behavioral variations are expressed in standard methods, compiled in anonymous classes.

\textbf{Reusability: Partial.} As an anonymous subclass is derived from the original class of an object, only instances of that same class can be migrated to the anonymous subclass. To apply the same instrumentation to an instance of another class, a new anonymous subclass must be created and the instrumented behavior must be recompiled in that subclass.

\textbf{Flexibility: None.} Instrumented methods must always be copied down to anonymous subclasses, and instrumentation must be inserted in the duplicated code. Without any tool built on top, that instrumentation is fully manual. Anonymous subclasses cannot be composed.

\textbf{Granularity: Method.} Instrumentation is implemented by recompiling modified copies of methods in anonymous subclasses. Sub-method level is achieved through manual rewriting of the method.

\textbf{Integration: Partial.} The object is migrated to an anonymous subclass, which does not break system tools. However, it is explicit that the object is now instance of an anonymous subclass. It may also break libraries and tools that use classes and class names as a discriminator.

\textbf{\textcode{Self} problem.} Solved by design: \textcode{self} always references the original object.

\textbf{\textcode{Super} problem.} Not solved. There are no means to express how to resolve the lookup when a message is sent to \textcode{super} from a method copied down in an anonymous subclass.
\section{Other documentation}
The Pharo Mooc provides materials on object-centric instrumentation based on object class migration and its flavours:

\begin{itemize}
\item http://rmod-pharo-mooc.lille.inria.fr/MOOC/Slides/Week7/C019-W7S04-OtherReflective.pdf
\item http://rmod-pharo-mooc.lille.inria.fr/MOOC/WebPortal/co/content\_78.html
\end{itemize}



\bibliographystyle{alpha}
\bibliography{book.bib}

% lulu requires an empty page at the end. That's why I'm using
% \backmatter here.
\backmatter

% Index would go here

\end{document}
