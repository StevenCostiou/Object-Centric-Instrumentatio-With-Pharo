% -*- mode: latex; -*- mustache tags:  
\documentclass[10pt,twoside,english]{_support/latex/sbabook/sbabook}
\let\wholebook=\relax

\usepackage{import}
\subimport{_support/latex/}{common.tex}

%=================================================================
% Debug packages for page layout and overfull lines
% Remove the showtrims document option before printing
\ifshowtrims
  \usepackage{showframe}
  \usepackage[color=magenta,width=5mm]{_support/latex/overcolored}
\fi


% =================================================================
\title{Object-Centric Instrumentation with Pharo}
\author{Steven Costiou}
\series{Square Bracket tutorials}

\hypersetup{
  pdftitle = {Object-Centric Instrumentation with Pharo},
  pdfauthor = {Steven Costiou},
  pdfkeywords = {object-centric, meta-programming}
}


% =================================================================
\begin{document}

% Title page and colophon on verso
\maketitle
\pagestyle{titlingpage}
\thispagestyle{titlingpage} % \pagestyle does not work on the first one…

\cleartoverso
{\small

  Copyright 2017 by Steven Costiou.

  The contents of this book are protected under the Creative Commons
  Attribution-ShareAlike 3.0 Unported license.

  You are \textbf{free}:
  \begin{itemize}
  \item to \textbf{Share}: to copy, distribute and transmit the work,
  \item to \textbf{Remix}: to adapt the work,
  \end{itemize}

  Under the following conditions:
  \begin{description}
  \item[Attribution.] You must attribute the work in the manner specified by the
    author or licensor (but not in any way that suggests that they endorse you
    or your use of the work).
  \item[Share Alike.] If you alter, transform, or build upon this work, you may
    distribute the resulting work only under the same, similar or a compatible
    license.
  \end{description}

  For any reuse or distribution, you must make clear to others the
  license terms of this work. The best way to do this is with a link to
  this web page: \\
  \url{http://creativecommons.org/licenses/by-sa/3.0/}

  Any of the above conditions can be waived if you get permission from
  the copyright holder. Nothing in this license impairs or restricts the
  author's moral rights.

  \begin{center}
    \includegraphics[width=0.2\textwidth]{_support/latex/sbabook/CreativeCommons-BY-SA.pdf}
  \end{center}

  Your fair dealing and other rights are in no way affected by the
  above. This is a human-readable summary of the Legal Code (the full
  license): \\
  \url{http://creativecommons.org/licenses/by-sa/3.0/legalcode}

  \vfill

  % Publication info would go here (publisher, ISBN, cover design…)
  Layout and typography based on the \textcode{sbabook} \LaTeX{} class by Damien
  Pollet.
}


\frontmatter
\pagestyle{plain}

\tableofcontents*
\clearpage\listoffigures

\mainmatter

\chapter{Talents}
Talents are originally behavioral units, that can be attached to an object to add, remove or alter behavior \cite{ressia2014talents}. Only the object to which a talent is attached is affected by behavioral variations. The latest talent implementation relies on trait definition and anonymous subclasses. Talents can be considered as object-centric, stateful-traits.
\section{Example}
Talents are based on traits. Objects can answer to the \textcode{\#addTalent:} messages (line 9), which takes a \textcode{Trait} as parameter. All behavior defined in the trait is flattened in the object. In the following illustration, we instantiate an anonymous trait (line 3), and we compile a method in this trait (line 4-8). That method is an instrumented version of the original \textcode{name} method of the class \textcode{Person}. This new method replaces the original one, until the talent is removed from the object (line 10). Talents now relies on anonymous subclasses, to which behavior is flattened before objects are migrated to the anonymous class.

\begin{displaycode}{plain}
|person talent|
  person := Person new.
	talent := Trait new.
	talent
		compile:
			'name: aName
	       self tag: aName.
	       name := aName'.
	person addTalent: talent. "adds the talent to the object"
  person removeTalent: talent. "removes the talent from the object"
\end{displaycode}
\section{Evaluation}
\textbf{Manipulated entity: Trait.} Behavioral variations are expressed using traits. It can be Traits defined in the image or anonymous trait instances in which specific behavior is manually compiled by the developer.

\textbf{Reusability: Yes.} A trait can be added as a Talent to any number of objects.

\textbf{Flexibility: Partial.} Using anonymous traits forces the user to manually compile code in the method. This is however necessary to achieve a sub-method granularity. Conflicts must be resolved manually when Traits are composed.

\textbf{Granularity: Method.} Traits add, remove or alter (through aliasing) the behavior of a method. It can be done at a sub-method level (\textit{e.g.} inserting a statement in the body of a method), but that requires manual rewriting of the method in the Trait.

\textbf{Integration: Partial.} The object is migrated to an anonymous subclass, which does not break system tools. However, it may break libraries that uses classes and class names as a discriminator.

\textbf{\textcode{Self} problem: Solved.} By design, \textcode{self} always references the original object.

\textbf{\textcode{Super} problem: Solved.} By flattening all methods that should be found in the super class into the anonymous subclass, and by replacing message sends to \textcode{super} by message sends to \textcode{self}.
\section{Other documentation}
The new implementation of Talents is available and documented on Github:

\begin{itemize}
\item https://github.com/tesonep/pharo-talents
\end{itemize}



\bibliographystyle{alpha}
\bibliography{book.bib}

% lulu requires an empty page at the end. That's why I'm using
% \backmatter here.
\backmatter

% Index would go here

\end{document}
