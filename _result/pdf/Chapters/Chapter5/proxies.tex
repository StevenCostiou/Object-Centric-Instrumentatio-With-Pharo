% -*- mode: latex; -*- mustache tags:  
\documentclass[10pt,twoside,english]{_support/latex/sbabook/sbabook}
\let\wholebook=\relax

\usepackage{import}
\subimport{_support/latex/}{common.tex}

%=================================================================
% Debug packages for page layout and overfull lines
% Remove the showtrims document option before printing
\ifshowtrims
  \usepackage{showframe}
  \usepackage[color=magenta,width=5mm]{_support/latex/overcolored}
\fi


% =================================================================
\title{Object-Centric Instrumentation with Pharo}
\author{Steven Costiou}
\series{Square Bracket tutorials}

\hypersetup{
  pdftitle = {Object-Centric Instrumentation with Pharo},
  pdfauthor = {Steven Costiou},
  pdfkeywords = {object-centric, meta-programming}
}


% =================================================================
\begin{document}

% Title page and colophon on verso
\maketitle
\pagestyle{titlingpage}
\thispagestyle{titlingpage} % \pagestyle does not work on the first one…

\cleartoverso
{\small

  Copyright 2017 by Steven Costiou.

  The contents of this book are protected under the Creative Commons
  Attribution-ShareAlike 3.0 Unported license.

  You are \textbf{free}:
  \begin{itemize}
  \item to \textbf{Share}: to copy, distribute and transmit the work,
  \item to \textbf{Remix}: to adapt the work,
  \end{itemize}

  Under the following conditions:
  \begin{description}
  \item[Attribution.] You must attribute the work in the manner specified by the
    author or licensor (but not in any way that suggests that they endorse you
    or your use of the work).
  \item[Share Alike.] If you alter, transform, or build upon this work, you may
    distribute the resulting work only under the same, similar or a compatible
    license.
  \end{description}

  For any reuse or distribution, you must make clear to others the
  license terms of this work. The best way to do this is with a link to
  this web page: \\
  \url{http://creativecommons.org/licenses/by-sa/3.0/}

  Any of the above conditions can be waived if you get permission from
  the copyright holder. Nothing in this license impairs or restricts the
  author's moral rights.

  \begin{center}
    \includegraphics[width=0.2\textwidth]{_support/latex/sbabook/CreativeCommons-BY-SA.pdf}
  \end{center}

  Your fair dealing and other rights are in no way affected by the
  above. This is a human-readable summary of the Legal Code (the full
  license): \\
  \url{http://creativecommons.org/licenses/by-sa/3.0/legalcode}

  \vfill

  % Publication info would go here (publisher, ISBN, cover design…)
  Layout and typography based on the \textcode{sbabook} \LaTeX{} class by Damien
  Pollet.
}


\frontmatter
\pagestyle{plain}

\tableofcontents*
\clearpage\listoffigures

\mainmatter

\chapter{Ghost}
Ghost is a general and uniform proxy implementation\cite{Mart14z}. A proxy replaces an object to control access to that object \cite{alpert1998design}. Object-centric instrumentation by means of proxies is done by swapping an object (and all its references) with a proxy object (and references to that proxy object). A proxy object is instance of a proxy class, in which access control is defined. Access control is generally implemented through a single interface, which is called each time a message is intercepted by the proxy. Control behavior then decides what to do with the received message.
\section{Example}
In this example, we use the original implementation of Ghost \cite{Mart14z}.

\begin{displaycode}{plain}
GHProxyHandler subclass: #MyProxyHandler
	instanceVariableNames: ''
	classVariableNames: ''
	package: 'ObjectCentricEvaluationExamples'

MyProxyHandler>>manageMessage: interception
	| message proxy target result |
	message := interception message.
	proxy := interception proxy.
	target := proxy proxyTarget.
	message selector == #name:
		ifTrue: [ target tag: message arguments first ].
	result := message sendTo: target.
	^ result == target
		ifTrue: [ proxy ]
		ifFalse: [ result ]

MyProxyHandler>>handleUninstall: anInterception
	^ anInterception proxy proxyTarget become: anInterception proxy.
\end{displaycode}

Then we instrument the objects

\begin{displaycode}{plain}
|person|
  person := Person new.
  GHTargetBasedProxy createProxyAndReplace: person handler: MyProxyHandler new.
  person uninstall
\end{displaycode}
\section{Evaluation}
\textbf{Manipulated entity: Classes.} Proxies are defined and/or configured in classes which inherits from Ghost internal classes. Typically, developers subclass the base message handler from Ghost to create a proxy model that implements the wanted instrumentation. An API is provided to apply a proxy to objects.

\textbf{Reusability: Complete.} The same proxy model can be reused to instrument any kind of object with the same instrumentation. Although Ghost proxies are not meant to be composed, it should be possible to \textit{proxify} a proxy, if a proper model is implemented to instrument a proxy object.

\textbf{Flexibility: Partial.} Because the user has to define a proxy model which specifies which messages are handled and/or which are not. However that allows developers to fully and transparently integrate the instrumented object into the environment.

\textbf{Granularity: Method.} A proxy intercepts messages sends to the object it \textit{proxifies}. It can execute instrumentation behavior before, after or instead the intercepted message. Sub-method instrumentation cannot be achieved by means of proxies.

\textbf{Integration: Full.} because meta-messages can bde efined

\textbf{\textcode{Self} problem: Implementation dependent.}

\textbf{\textcode{Super} problem: Implementation dependent.} ?
\section{Other documentation}
Implementations and documentation based on the original Ghost paper \cite{Mart14z}:

\begin{itemize}
\item http://esug.org/data/ESUG2011/IWST/PRESENTATIONS/23.Mariano\_Peck-Ghost-ESUG2011.pdf
\item https://rmod.inria.fr/archives/papers/Mart14z-Ghost-Final.pdf
\item https://gitlab.inria.fr/RMOD/Ghost
\item https://github.com/guillep/avatar
\end{itemize}

Another implementation of Ghost:

\begin{itemize}
\item https://github.com/pharo-ide/Ghost
\item http://dionisiydk.blogspot.com/2016/04/halt-next-object-call.html
\end{itemize}



\bibliographystyle{alpha}
\bibliography{book.bib}

% lulu requires an empty page at the end. That's why I'm using
% \backmatter here.
\backmatter

% Index would go here

\end{document}
