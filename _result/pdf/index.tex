% -*- mode: latex; -*- mustache tags:  
\documentclass[10pt,twoside,english]{_support/latex/sbabook/sbabook}
\let\wholebook=\relax

\usepackage{import}
\subimport{_support/latex/}{common.tex}

%=================================================================
% Debug packages for page layout and overfull lines
% Remove the showtrims document option before printing
\ifshowtrims
  \usepackage{showframe}
  \usepackage[color=magenta,width=5mm]{_support/latex/overcolored}
\fi


% =================================================================
\title{Object-Centric Instrumentation with Pharo}
\author{Steven Costiou}
\series{Square Bracket tutorials}

\hypersetup{
  pdftitle = {Object-Centric Instrumentation with Pharo},
  pdfauthor = {Steven Costiou},
  pdfkeywords = {object-centric, meta-programming}
}


% =================================================================
\begin{document}

% Title page and colophon on verso
\maketitle
\pagestyle{titlingpage}
\thispagestyle{titlingpage} % \pagestyle does not work on the first one…

\cleartoverso
{\small

  Copyright 2017 by Steven Costiou.

  The contents of this book are protected under the Creative Commons
  Attribution-ShareAlike 3.0 Unported license.

  You are \textbf{free}:
  \begin{itemize}
  \item to \textbf{Share}: to copy, distribute and transmit the work,
  \item to \textbf{Remix}: to adapt the work,
  \end{itemize}

  Under the following conditions:
  \begin{description}
  \item[Attribution.] You must attribute the work in the manner specified by the
    author or licensor (but not in any way that suggests that they endorse you
    or your use of the work).
  \item[Share Alike.] If you alter, transform, or build upon this work, you may
    distribute the resulting work only under the same, similar or a compatible
    license.
  \end{description}

  For any reuse or distribution, you must make clear to others the
  license terms of this work. The best way to do this is with a link to
  this web page: \\
  \url{http://creativecommons.org/licenses/by-sa/3.0/}

  Any of the above conditions can be waived if you get permission from
  the copyright holder. Nothing in this license impairs or restricts the
  author's moral rights.

  \begin{center}
    \includegraphics[width=0.2\textwidth]{_support/latex/sbabook/CreativeCommons-BY-SA.pdf}
  \end{center}

  Your fair dealing and other rights are in no way affected by the
  above. This is a human-readable summary of the Legal Code (the full
  license): \\
  \url{http://creativecommons.org/licenses/by-sa/3.0/legalcode}

  \vfill

  % Publication info would go here (publisher, ISBN, cover design…)
  Layout and typography based on the \textcode{sbabook} \LaTeX{} class by Damien
  Pollet.
}


\frontmatter
\pagestyle{plain}

\tableofcontents*
\clearpage\listoffigures

\mainmatter

\chapter{Introduction}
This booklet is about object-centric instrumentation in Pharo. An instrumentation is object-centric if it applies to one specific object (or a set of objects), without consideration of its class. It means the instrumentation can be applied on one object, leaving untouched all other instances of its class, or to an heterogeneous set of instances of different classes. This booklet gives an overview of available object-centric instrumentation techniques in Pharo, either present in the standard distribution or available on download. We will not go into deep technical usage description, nor into implementation details. Each chapter illustrates one solution with examples, and gives the necessary references if one wants to go deeper in the study of the solution. We study each technique following a three-fold evaluation. First, the studied technique is applied on a simple example of object-centric instrumentation. Second, the technique is evaluated against a set of desirable properties. Finally, performance overhead are evaluated. Only the raw solution is evaluated, without considering the possibility of enhancing the technique by building something on top.
\section{Illustration example}
Each studied solution is experimented on a trivial example of object-centric behavior instrumentation. This example is illustrated in script \ref{motivation-example}. Two instances of \textcode{OrderedCollection} are created, and to each of these instances is sent the \textcode{add:} message with a string as a parameter. The instrumentation must happen in-between. We want to instrument the \textcode{col1} object, so that when the \textcode{add:} message is received, the size of the collection and the added object (passed as parameter) are printed in the \textcode{Transcript}. The \textcode{col2} object must remain unnaffected.

\begin{listing}[float, label=motivation-example]{smalltalk}{Trivial example for object-centric instrumentation}
|col1 col2|
col1 := OrderedCollection new.
col2 := OrderedCollection new.

"...instrumentation must happen here..."

col1 add: 'Hello World'.
col2 add: 'Hello World'.
\end{listing}
\section{Evaluation criteria}
Each solution is evaluated against the following desirable properties.

\begin{tabular}{ll}
\toprule
\textbf{Property} & \textbf{Definition} \\
\midrule
Manipulated entity & The unit of instrumentation \\
 & (\textit{e.g.} a class, a Trait, an object...) \\
Reusability & The entity can be reused to instrument different objects \\
Flexibility & Instrumentation does not put constraint on the \\
 & source code or in the coding style \\
Granularity & The level of at which behavior can be instrumented \\
 & (\textit{e.g.} method, AST...) \\
Integration & Instrumentation does not break system features \\
\bottomrule
\end{tabular}
\section{Performance overhead evaluation}
To provide a approximation of the performance overhead due to instrumentation, we compare the execution time of a block of code without instrumentation with the execution time of an instrumented block of code. Script \ref{performance-overhead-method} shows how the average execution time is computed. An ordered collection is instantiated (\textcode{collection}) and the \textcode{add:} message is sent 1000 times to that object. Before actually sending that message to the collection, the parameter is logged in the Transcript. Each time, the execution time of the parameter logging and the \textcode{add:} method is stored into the \textcode{execTime} collection. The average execution time is computed from the stored results. One reference evaluation is performed without instrumentation. This reference average time is used for comparison with instrumented versions evaluations. For each studied technique, the same code is executed, but with an object-centric instrumentation put on the \textcode{collection} object. However, that time the logging instruction of the \textcode{add:} mesage parameter is removed from the script, and added as an object-centric instrumentation on the \textcode{add:} method of the collection object. We suppose that the execution time of this instruction is constant, so that we can isolate the cost of the instrumentation mechanism and compare it against the cost of having the instrumentation code hard-coded.

\begin{listing}[float, label=performance-overhead-method]{smalltalk}{Performance overhead evaluation script}
|execTimes collection|
execTimes := OrderedCollection new.
collection := OrderedCollection new.
1 to: 1000 do:[:i|
  execTimes add: [
    i logCr. "That line is removed for instrumented evaluations"
    collection add: i] timeToRun].
execTimes average
\end{listing}
\section{Structure of the book}
The second chapter will provide an overview of the evaluation results of object-centric instrumentation techniques available in Pharo. A reader may directly read this chapter if he is already familiar with the Pharo techniques presented in the book. Chapters 3 to 7 describe five solutions for object-centric instrumentation, and provide an evaluation of these solutions. Chapter 8 drafts the premises of an object-centric debugger and concludes the book.

\chapter{Summary of the overall evaluations}
If you already know Pharo and (some of) the presented technique, this chapter is a global summary with spoilers.

\chapter{Anonymous subclasses}\section{What are Talents}\section{Example}\subsection{Installing Talents}
aa

\begin{listing}[float, label=install]{smalltalk}{Installation from Github}

Metacello new
  baseline: 'Talents';
  repository: 'github://tesonep/pharo-talents/src';
  load.
\end{listing}
\subsection{Example}
bb

\begin{listing}[float, label=talent-example]{smalltalk}{Installation from Github}

talent := Trait named: 'MyTalent'.
talent compile: 'add: anObject
anObject logCr.
super add: anObject'.
col := OrderedCollection new.
col addTalent: talent.
col add: 'This is an added object.'
\end{listing}
\section{Evaluation}
cc

\begin{note}
this is a note annotation.
\end{note}

\begin{todo}
this is a todo annotation
\end{todo}

\begin{tabular}{ll}
\toprule
\textbf{Country} & \textbf{Capital} \\
\midrule
France & Paris \\
Belgium & Brussels \\
\textbf{Country} & \textbf{Capital} \\
\midrule
France & Paris \\
Belgium & Brussels \\
\bottomrule
\end{tabular}
\chapter{Talents}
Talents \cite{ressia2014talents}\textcode{ }is this.
\section{What are Talents}\section{Example}\subsection{Installing Talents}
aa

\begin{listing}[float, label=install]{smalltalk}{Installation from Github}

Metacello new
  baseline: 'Talents';
  repository: 'github://tesonep/pharo-talents/src';
  load.
\end{listing}
\subsection{Example}
bb

\begin{listing}[float, label=talent-example]{smalltalk}{Installation from Github}

talent := Trait named: 'MyTalent'.
talent compile: 'add: anObject
anObject logCr.
super add: anObject'.
col := OrderedCollection new.
col addTalent: talent.
col add: 'This is an added object.'
\end{listing}
\section{Evaluation}
cc

\begin{note}
this is a note annotation.
\end{note}

\begin{todo}
this is a todo annotation
\end{todo}

\begin{tabular}{ll}
\toprule
\textbf{Country} & \textbf{Capital} \\
\midrule
France & Paris \\
Belgium & Brussels \\
\textbf{Country} & \textbf{Capital} \\
\midrule
France & Paris \\
Belgium & Brussels \\
\bottomrule
\end{tabular}
\chapter{Proxies}\section{What are Talents}\section{Example}\subsection{Installing Talents}
aa

\begin{listing}[float, label=install]{smalltalk}{Installation from Github}

Metacello new
  baseline: 'Talents';
  repository: 'github://tesonep/pharo-talents/src';
  load.
\end{listing}
\subsection{Example}
bb

\begin{listing}[float, label=talent-example]{smalltalk}{Installation from Github}

talent := Trait named: 'MyTalent'.
talent compile: 'add: anObject
anObject logCr.
super add: anObject'.
col := OrderedCollection new.
col addTalent: talent.
col add: 'This is an added object.'
\end{listing}
\section{Evaluation}
cc

\begin{note}
this is a note annotation.
\end{note}

\begin{todo}
this is a todo annotation
\end{todo}

\begin{tabular}{ll}
\toprule
\textbf{Country} & \textbf{Capital} \\
\midrule
France & Paris \\
Belgium & Brussels \\
\textbf{Country} & \textbf{Capital} \\
\midrule
France & Paris \\
Belgium & Brussels \\
\bottomrule
\end{tabular}
\chapter{Reflectivity}
Talents \cite{ressia2014talents}is this.
\section{What are Talents}\section{Example}\subsection{Installing Talents}
aa

\begin{listing}[float, label=install]{smalltalk}{Installation from Github}

Metacello new
  baseline: 'Talents';
  repository: 'github://tesonep/pharo-talents/src';
  load.
\end{listing}
\subsection{Example}
bb

\begin{listing}[float, label=talent-example]{smalltalk}{Installation from Github}

talent := Trait named: 'MyTalent'.
talent compile: 'add: anObject
anObject logCr.
super add: anObject'.
col := OrderedCollection new.
col addTalent: talent.
col add: 'This is an added object.'
\end{listing}
\section{Evaluation}
cc

\begin{note}
this is a note annotation.
\end{note}

\begin{todo}
this is a todo annotation
\end{todo}

\begin{tabular}{ll}
\toprule
\textbf{Country} & \textbf{Capital} \\
\midrule
France & Paris \\
Belgium & Brussels \\
\textbf{Country} & \textbf{Capital} \\
\midrule
France & Paris \\
Belgium & Brussels \\
\bottomrule
\end{tabular}
\chapter{Low-level techniques}\section{Example}\subsection{Installing Talents}
aa

\begin{listing}[float, label=install]{smalltalk}{Installation from Github}

Metacello new
  baseline: 'Talents';
  repository: 'github://tesonep/pharo-talents/src';
  load.
\end{listing}
\subsection{Example}
bb

\begin{listing}[float, label=talent-example]{smalltalk}{Installation from Github}

talent := Trait named: 'MyTalent'.
talent compile: 'add: anObject
anObject logCr.
super add: anObject'.
col := OrderedCollection new.
col addTalent: talent.
col add: 'This is an added object.'
\end{listing}
\section{Evaluation}
cc

\begin{note}
this is a note annotation.
\end{note}

\begin{todo}
this is a todo annotation
\end{todo}

\begin{tabular}{ll}
\toprule
\textbf{Country} & \textbf{Capital} \\
\midrule
France & Paris \\
Belgium & Brussels \\
\textbf{Country} & \textbf{Capital} \\
\midrule
France & Paris \\
Belgium & Brussels \\
\bottomrule
\end{tabular}
\chapter{Conclusion}\section{Example}\subsection{Installing Talents}
aa

\begin{listing}[float, label=install]{smalltalk}{Installation from Github}

Metacello new
  baseline: 'Talents';
  repository: 'github://tesonep/pharo-talents/src';
  load.
\end{listing}
\subsection{Example}
bb

\begin{listing}[float, label=talent-example]{smalltalk}{Installation from Github}

talent := Trait named: 'MyTalent'.
talent compile: 'add: anObject
anObject logCr.
super add: anObject'.
col := OrderedCollection new.
col addTalent: talent.
col add: 'This is an added object.'
\end{listing}
\section{Evaluation}
cc

\begin{note}
this is a note annotation.
\end{note}

\begin{todo}
this is a todo annotation
\end{todo}

\begin{tabular}{ll}
\toprule
\textbf{Country} & \textbf{Capital} \\
\midrule
France & Paris \\
Belgium & Brussels \\
\textbf{Country} & \textbf{Capital} \\
\midrule
France & Paris \\
Belgium & Brussels \\
\bottomrule
\end{tabular}



\bibliographystyle{alpha}
\bibliography{book.bib}

% lulu requires an empty page at the end. That's why I'm using
% \backmatter here.
\backmatter

% Index would go here

\end{document}
